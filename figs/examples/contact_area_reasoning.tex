% Options for packages loaded elsewhere
\PassOptionsToPackage{unicode}{hyperref}
\PassOptionsToPackage{hyphens}{url}
\PassOptionsToPackage{dvipsnames,svgnames,x11names}{xcolor}
%
\documentclass[
  letterpaper,
  DIV=11,
  numbers=noendperiod]{scrartcl}

\usepackage{amsmath,amssymb}
\usepackage{iftex}
\ifPDFTeX
  \usepackage[T1]{fontenc}
  \usepackage[utf8]{inputenc}
  \usepackage{textcomp} % provide euro and other symbols
\else % if luatex or xetex
  \usepackage{unicode-math}
  \defaultfontfeatures{Scale=MatchLowercase}
  \defaultfontfeatures[\rmfamily]{Ligatures=TeX,Scale=1}
\fi
\usepackage{lmodern}
\ifPDFTeX\else  
    % xetex/luatex font selection
\fi
% Use upquote if available, for straight quotes in verbatim environments
\IfFileExists{upquote.sty}{\usepackage{upquote}}{}
\IfFileExists{microtype.sty}{% use microtype if available
  \usepackage[]{microtype}
  \UseMicrotypeSet[protrusion]{basicmath} % disable protrusion for tt fonts
}{}
\makeatletter
\@ifundefined{KOMAClassName}{% if non-KOMA class
  \IfFileExists{parskip.sty}{%
    \usepackage{parskip}
  }{% else
    \setlength{\parindent}{0pt}
    \setlength{\parskip}{6pt plus 2pt minus 1pt}}
}{% if KOMA class
  \KOMAoptions{parskip=half}}
\makeatother
\usepackage{xcolor}
\setlength{\emergencystretch}{3em} % prevent overfull lines
\setcounter{secnumdepth}{-\maxdimen} % remove section numbering
% Make \paragraph and \subparagraph free-standing
\makeatletter
\ifx\paragraph\undefined\else
  \let\oldparagraph\paragraph
  \renewcommand{\paragraph}{
    \@ifstar
      \xxxParagraphStar
      \xxxParagraphNoStar
  }
  \newcommand{\xxxParagraphStar}[1]{\oldparagraph*{#1}\mbox{}}
  \newcommand{\xxxParagraphNoStar}[1]{\oldparagraph{#1}\mbox{}}
\fi
\ifx\subparagraph\undefined\else
  \let\oldsubparagraph\subparagraph
  \renewcommand{\subparagraph}{
    \@ifstar
      \xxxSubParagraphStar
      \xxxSubParagraphNoStar
  }
  \newcommand{\xxxSubParagraphStar}[1]{\oldsubparagraph*{#1}\mbox{}}
  \newcommand{\xxxSubParagraphNoStar}[1]{\oldsubparagraph{#1}\mbox{}}
\fi
\makeatother

\usepackage{color}
\usepackage{fancyvrb}
\newcommand{\VerbBar}{|}
\newcommand{\VERB}{\Verb[commandchars=\\\{\}]}
\DefineVerbatimEnvironment{Highlighting}{Verbatim}{commandchars=\\\{\}}
% Add ',fontsize=\small' for more characters per line
\usepackage{framed}
\definecolor{shadecolor}{RGB}{241,243,245}
\newenvironment{Shaded}{\begin{snugshade}}{\end{snugshade}}
\newcommand{\AlertTok}[1]{\textcolor[rgb]{0.68,0.00,0.00}{#1}}
\newcommand{\AnnotationTok}[1]{\textcolor[rgb]{0.37,0.37,0.37}{#1}}
\newcommand{\AttributeTok}[1]{\textcolor[rgb]{0.40,0.45,0.13}{#1}}
\newcommand{\BaseNTok}[1]{\textcolor[rgb]{0.68,0.00,0.00}{#1}}
\newcommand{\BuiltInTok}[1]{\textcolor[rgb]{0.00,0.23,0.31}{#1}}
\newcommand{\CharTok}[1]{\textcolor[rgb]{0.13,0.47,0.30}{#1}}
\newcommand{\CommentTok}[1]{\textcolor[rgb]{0.37,0.37,0.37}{#1}}
\newcommand{\CommentVarTok}[1]{\textcolor[rgb]{0.37,0.37,0.37}{\textit{#1}}}
\newcommand{\ConstantTok}[1]{\textcolor[rgb]{0.56,0.35,0.01}{#1}}
\newcommand{\ControlFlowTok}[1]{\textcolor[rgb]{0.00,0.23,0.31}{\textbf{#1}}}
\newcommand{\DataTypeTok}[1]{\textcolor[rgb]{0.68,0.00,0.00}{#1}}
\newcommand{\DecValTok}[1]{\textcolor[rgb]{0.68,0.00,0.00}{#1}}
\newcommand{\DocumentationTok}[1]{\textcolor[rgb]{0.37,0.37,0.37}{\textit{#1}}}
\newcommand{\ErrorTok}[1]{\textcolor[rgb]{0.68,0.00,0.00}{#1}}
\newcommand{\ExtensionTok}[1]{\textcolor[rgb]{0.00,0.23,0.31}{#1}}
\newcommand{\FloatTok}[1]{\textcolor[rgb]{0.68,0.00,0.00}{#1}}
\newcommand{\FunctionTok}[1]{\textcolor[rgb]{0.28,0.35,0.67}{#1}}
\newcommand{\ImportTok}[1]{\textcolor[rgb]{0.00,0.46,0.62}{#1}}
\newcommand{\InformationTok}[1]{\textcolor[rgb]{0.37,0.37,0.37}{#1}}
\newcommand{\KeywordTok}[1]{\textcolor[rgb]{0.00,0.23,0.31}{\textbf{#1}}}
\newcommand{\NormalTok}[1]{\textcolor[rgb]{0.00,0.23,0.31}{#1}}
\newcommand{\OperatorTok}[1]{\textcolor[rgb]{0.37,0.37,0.37}{#1}}
\newcommand{\OtherTok}[1]{\textcolor[rgb]{0.00,0.23,0.31}{#1}}
\newcommand{\PreprocessorTok}[1]{\textcolor[rgb]{0.68,0.00,0.00}{#1}}
\newcommand{\RegionMarkerTok}[1]{\textcolor[rgb]{0.00,0.23,0.31}{#1}}
\newcommand{\SpecialCharTok}[1]{\textcolor[rgb]{0.37,0.37,0.37}{#1}}
\newcommand{\SpecialStringTok}[1]{\textcolor[rgb]{0.13,0.47,0.30}{#1}}
\newcommand{\StringTok}[1]{\textcolor[rgb]{0.13,0.47,0.30}{#1}}
\newcommand{\VariableTok}[1]{\textcolor[rgb]{0.07,0.07,0.07}{#1}}
\newcommand{\VerbatimStringTok}[1]{\textcolor[rgb]{0.13,0.47,0.30}{#1}}
\newcommand{\WarningTok}[1]{\textcolor[rgb]{0.37,0.37,0.37}{\textit{#1}}}

\providecommand{\tightlist}{%
  \setlength{\itemsep}{0pt}\setlength{\parskip}{0pt}}\usepackage{longtable,booktabs,array}
\usepackage{calc} % for calculating minipage widths
% Correct order of tables after \paragraph or \subparagraph
\usepackage{etoolbox}
\makeatletter
\patchcmd\longtable{\par}{\if@noskipsec\mbox{}\fi\par}{}{}
\makeatother
% Allow footnotes in longtable head/foot
\IfFileExists{footnotehyper.sty}{\usepackage{footnotehyper}}{\usepackage{footnote}}
\makesavenoteenv{longtable}
\usepackage{graphicx}
\makeatletter
\def\maxwidth{\ifdim\Gin@nat@width>\linewidth\linewidth\else\Gin@nat@width\fi}
\def\maxheight{\ifdim\Gin@nat@height>\textheight\textheight\else\Gin@nat@height\fi}
\makeatother
% Scale images if necessary, so that they will not overflow the page
% margins by default, and it is still possible to overwrite the defaults
% using explicit options in \includegraphics[width, height, ...]{}
\setkeys{Gin}{width=\maxwidth,height=\maxheight,keepaspectratio}
% Set default figure placement to htbp
\makeatletter
\def\fps@figure{htbp}
\makeatother

\KOMAoption{captions}{tableheading}
\makeatletter
\@ifpackageloaded{caption}{}{\usepackage{caption}}
\AtBeginDocument{%
\ifdefined\contentsname
  \renewcommand*\contentsname{Table of contents}
\else
  \newcommand\contentsname{Table of contents}
\fi
\ifdefined\listfigurename
  \renewcommand*\listfigurename{List of Figures}
\else
  \newcommand\listfigurename{List of Figures}
\fi
\ifdefined\listtablename
  \renewcommand*\listtablename{List of Tables}
\else
  \newcommand\listtablename{List of Tables}
\fi
\ifdefined\figurename
  \renewcommand*\figurename{Figure}
\else
  \newcommand\figurename{Figure}
\fi
\ifdefined\tablename
  \renewcommand*\tablename{Table}
\else
  \newcommand\tablename{Table}
\fi
}
\@ifpackageloaded{float}{}{\usepackage{float}}
\floatstyle{ruled}
\@ifundefined{c@chapter}{\newfloat{codelisting}{h}{lop}}{\newfloat{codelisting}{h}{lop}[chapter]}
\floatname{codelisting}{Listing}
\newcommand*\listoflistings{\listof{codelisting}{List of Listings}}
\makeatother
\makeatletter
\makeatother
\makeatletter
\@ifpackageloaded{caption}{}{\usepackage{caption}}
\@ifpackageloaded{subcaption}{}{\usepackage{subcaption}}
\makeatother

\ifLuaTeX
  \usepackage{selnolig}  % disable illegal ligatures
\fi
\usepackage{bookmark}

\IfFileExists{xurl.sty}{\usepackage{xurl}}{} % add URL line breaks if available
\urlstyle{same} % disable monospaced font for URLs
\hypersetup{
  pdftitle={Reasoning for Contact Area Model},
  pdfauthor={Alex Cebulski},
  colorlinks=true,
  linkcolor={blue},
  filecolor={Maroon},
  citecolor={Blue},
  urlcolor={Blue},
  pdfcreator={LaTeX via pandoc}}


\title{Reasoning for Contact Area Model}
\author{Alex Cebulski}
\date{}

\begin{document}
\maketitle


The reasoning for using \texttt{sin} as a factor for increased contact
area is as follows:

Let:

\begin{itemize}
\tightlist
\item
  \(w_p\) be the contact width contributing to interception, i.e., a
  plane perpendicular to the snowfall that contacts the canopy as it
  moves through the canopy space
\item
  \(h_c\) be the height of the canopy from the top down contacting
  snowfall
\item
  \(w_c\) be the width of the canopy at height \(h_c\)
\item
  \(A_c\) be the area of the canopy in contact with snowfall
\item
  \(\theta_h\) be the snowfall trajectory angle from zenith
\item
  \(h_t\) be the vertical height of the canopy
\item
  \(w_t\) be the base width of the canopy
\end{itemize}

Then:

\[
w_p = \text{sin}(\theta_h) * h_t
\]

\[
h_c = \text{sin}(\theta_h) * w_p
\]

\[
w_c = \frac{w_t}{h_t} * h_c
\]

\[
A_c = h_c*\frac{w_c}{2}*2
\]

A lightly more precise calculation of \(w_p\) which takes into account
the tilting of the contact plane would be:

Let:

\begin{itemize}
\tightlist
\item
  \(\theta_b\) be the angle of the bottom right of the tree from
  vertical.
\item
  \(h_s\) be the length of the side of the tree aka hypotenuse
\end{itemize}

Then:

\[
\theta_b = \text{atan}(\frac{w_t}{h_t}/h_t)
\]

\[
h_s = \sqrt{h_t^2 + (w_t/2)^2}
\]

\[
w_p = \text{sin}(\theta_b+\theta_h)*h_s
\]

\begin{Shaded}
\begin{Highlighting}[]
\CommentTok{\# Load ggplot2 library}
\FunctionTok{library}\NormalTok{(ggplot2)}

\CommentTok{\# Define parameters}
\NormalTok{h\_t }\OtherTok{\textless{}{-}} \DecValTok{20}
\NormalTok{w\_t }\OtherTok{\textless{}{-}} \DecValTok{5}
\NormalTok{theta\_h }\OtherTok{\textless{}{-}} \DecValTok{5} \SpecialCharTok{*}\NormalTok{ (pi}\SpecialCharTok{/}\DecValTok{180}\NormalTok{)}

\NormalTok{theta\_b }\OtherTok{\textless{}{-}} \FunctionTok{atan}\NormalTok{((w\_t}\SpecialCharTok{/}\DecValTok{2}\NormalTok{) }\SpecialCharTok{/}\NormalTok{ h\_t)}
\NormalTok{h\_s }\OtherTok{\textless{}{-}} \FunctionTok{sqrt}\NormalTok{(h\_t}\SpecialCharTok{\^{}}\DecValTok{2} \SpecialCharTok{+}\NormalTok{ (w\_t}\SpecialCharTok{/}\DecValTok{2}\NormalTok{)}\SpecialCharTok{\^{}}\DecValTok{2}\NormalTok{)}
\NormalTok{w\_p }\OtherTok{\textless{}{-}} \FunctionTok{sin}\NormalTok{(theta\_b }\SpecialCharTok{+}\NormalTok{ theta\_h) }\SpecialCharTok{*}\NormalTok{ h\_s}
\NormalTok{h\_c }\OtherTok{\textless{}{-}} \FunctionTok{sin}\NormalTok{(theta\_h) }\SpecialCharTok{*}\NormalTok{ (w\_p)}
\NormalTok{x }\OtherTok{\textless{}{-}}\NormalTok{ h\_c }\SpecialCharTok{/} \FunctionTok{tan}\NormalTok{(theta\_h)}

\CommentTok{\# Calculate w\_b (the top side of the blue triangle)}
\NormalTok{w\_b }\OtherTok{\textless{}{-}} \FunctionTok{sqrt}\NormalTok{((w\_t }\SpecialCharTok{/} \DecValTok{2}\NormalTok{)}\SpecialCharTok{\^{}}\DecValTok{2} \SpecialCharTok{+}\NormalTok{ h\_t}\SpecialCharTok{\^{}}\DecValTok{2}\NormalTok{)  }\CommentTok{\# Using Pythagoras to calculate the top side}

\CommentTok{\# Define the points of the green triangle}
\NormalTok{triangle\_data }\OtherTok{\textless{}{-}} \FunctionTok{data.frame}\NormalTok{(}
  \AttributeTok{x =} \FunctionTok{c}\NormalTok{(}\DecValTok{0}\NormalTok{, w\_t, w\_t }\SpecialCharTok{/} \DecValTok{2}\NormalTok{),}
  \AttributeTok{y =} \FunctionTok{c}\NormalTok{(}\DecValTok{0}\NormalTok{, }\DecValTok{0}\NormalTok{, h\_t),}
  \AttributeTok{label =} \FunctionTok{c}\NormalTok{(}\StringTok{"0"}\NormalTok{, }\StringTok{"w\_c"}\NormalTok{, }\StringTok{"h\_c"}\NormalTok{)}
\NormalTok{)}

\CommentTok{\# Define the points of the blue triangle}
\NormalTok{triangle\_data2 }\OtherTok{\textless{}{-}} \FunctionTok{data.frame}\NormalTok{(}
  \AttributeTok{x =} \FunctionTok{c}\NormalTok{(w\_t }\SpecialCharTok{/} \DecValTok{2}\NormalTok{, w\_t, (w\_t }\SpecialCharTok{/} \DecValTok{2}\NormalTok{) }\SpecialCharTok{+}\NormalTok{ x),}
  \AttributeTok{y =} \FunctionTok{c}\NormalTok{(h\_t, }\DecValTok{0}\NormalTok{, h\_t }\SpecialCharTok{{-}}\NormalTok{ h\_c),}
  \AttributeTok{label =} \FunctionTok{c}\NormalTok{(}\StringTok{"0"}\NormalTok{, }\StringTok{"w\_c"}\NormalTok{, }\StringTok{"h\_c"}\NormalTok{)}
\NormalTok{)}

\CommentTok{\# Plot the triangle with labels}
\FunctionTok{ggplot}\NormalTok{(triangle\_data, }\FunctionTok{aes}\NormalTok{(x, y)) }\SpecialCharTok{+}
  \FunctionTok{geom\_polygon}\NormalTok{(}\AttributeTok{fill =} \ConstantTok{NA}\NormalTok{, }\AttributeTok{color =} \StringTok{"green"}\NormalTok{, }\AttributeTok{size =} \DecValTok{1}\NormalTok{) }\SpecialCharTok{+}  \CommentTok{\# Create the green triangle}
  \FunctionTok{geom\_polygon}\NormalTok{(}\AttributeTok{data =}\NormalTok{ triangle\_data2, }\FunctionTok{aes}\NormalTok{(x, y), }\AttributeTok{colour =} \StringTok{"blue"}\NormalTok{, }\AttributeTok{fill =} \ConstantTok{NA}\NormalTok{) }\SpecialCharTok{+}  \CommentTok{\# Blue triangle}
  \FunctionTok{annotate}\NormalTok{(}\StringTok{"text"}\NormalTok{, }\AttributeTok{x =}\NormalTok{ (w\_t }\SpecialCharTok{/} \DecValTok{2} \SpecialCharTok{+}\NormalTok{ (w\_t }\SpecialCharTok{/} \DecValTok{2} \SpecialCharTok{+}\NormalTok{ x)) }\SpecialCharTok{/} \DecValTok{2}\NormalTok{, }\AttributeTok{y =}\NormalTok{ h\_t}\DecValTok{{-}1}\NormalTok{, }\AttributeTok{label =} \StringTok{"w\_p"}\NormalTok{, }\AttributeTok{size =} \DecValTok{4}\NormalTok{, }\AttributeTok{color =} \StringTok{"blue"}\NormalTok{) }\SpecialCharTok{+}
  \FunctionTok{labs}\NormalTok{(}\AttributeTok{x =} \StringTok{"Width (m)"}\NormalTok{, }\AttributeTok{y =} \StringTok{"Height (m)"}\NormalTok{) }\SpecialCharTok{+}
  \FunctionTok{coord\_fixed}\NormalTok{(}\AttributeTok{ratio =} \DecValTok{1}\NormalTok{)  }\CommentTok{\# Ensure the triangle\textquotesingle{}s proportions are fixed}
\end{Highlighting}
\end{Shaded}

\begin{verbatim}
Warning: Using `size` aesthetic for lines was deprecated in ggplot2 3.4.0.
i Please use `linewidth` instead.
\end{verbatim}

\includegraphics{contact_area_reasoning_files/figure-pdf/unnamed-chunk-1-1.pdf}

For small \(\theta_h\) we should also decrease \(w_p\) as the increase
on one side of the tree is balanced by the decrease in contacts on the
right side:

w\_p\_left \textless- sin(theta\_b - theta\_h)*hyp

which gives the contact length on the left side.




\end{document}
