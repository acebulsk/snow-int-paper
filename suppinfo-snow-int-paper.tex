% Options for packages loaded elsewhere
\PassOptionsToPackage{unicode}{hyperref}
\PassOptionsToPackage{hyphens}{url}
%
\documentclass[
  letterpaper,
  DIV=11,
  numbers=noendperiod]{scrartcl}

\usepackage{amsmath,amssymb}
\usepackage{setspace}
\usepackage{iftex}
\ifPDFTeX
  \usepackage[T1]{fontenc}
  \usepackage[utf8]{inputenc}
  \usepackage{textcomp} % provide euro and other symbols
\else % if luatex or xetex
  \usepackage{unicode-math}
  \defaultfontfeatures{Scale=MatchLowercase}
  \defaultfontfeatures[\rmfamily]{Ligatures=TeX,Scale=1}
\fi
\usepackage{lmodern}
\ifPDFTeX\else  
    % xetex/luatex font selection
\fi
% Use upquote if available, for straight quotes in verbatim environments
\IfFileExists{upquote.sty}{\usepackage{upquote}}{}
\IfFileExists{microtype.sty}{% use microtype if available
  \usepackage[]{microtype}
  \UseMicrotypeSet[protrusion]{basicmath} % disable protrusion for tt fonts
}{}
\makeatletter
\@ifundefined{KOMAClassName}{% if non-KOMA class
  \IfFileExists{parskip.sty}{%
    \usepackage{parskip}
  }{% else
    \setlength{\parindent}{0pt}
    \setlength{\parskip}{6pt plus 2pt minus 1pt}}
}{% if KOMA class
  \KOMAoptions{parskip=half}}
\makeatother
\usepackage{xcolor}
\setlength{\emergencystretch}{3em} % prevent overfull lines
\setcounter{secnumdepth}{5}
% Make \paragraph and \subparagraph free-standing
\makeatletter
\ifx\paragraph\undefined\else
  \let\oldparagraph\paragraph
  \renewcommand{\paragraph}{
    \@ifstar
      \xxxParagraphStar
      \xxxParagraphNoStar
  }
  \newcommand{\xxxParagraphStar}[1]{\oldparagraph*{#1}\mbox{}}
  \newcommand{\xxxParagraphNoStar}[1]{\oldparagraph{#1}\mbox{}}
\fi
\ifx\subparagraph\undefined\else
  \let\oldsubparagraph\subparagraph
  \renewcommand{\subparagraph}{
    \@ifstar
      \xxxSubParagraphStar
      \xxxSubParagraphNoStar
  }
  \newcommand{\xxxSubParagraphStar}[1]{\oldsubparagraph*{#1}\mbox{}}
  \newcommand{\xxxSubParagraphNoStar}[1]{\oldsubparagraph{#1}\mbox{}}
\fi
\makeatother


\providecommand{\tightlist}{%
  \setlength{\itemsep}{0pt}\setlength{\parskip}{0pt}}\usepackage{longtable,booktabs,array}
\usepackage{calc} % for calculating minipage widths
% Correct order of tables after \paragraph or \subparagraph
\usepackage{etoolbox}
\makeatletter
\patchcmd\longtable{\par}{\if@noskipsec\mbox{}\fi\par}{}{}
\makeatother
% Allow footnotes in longtable head/foot
\IfFileExists{footnotehyper.sty}{\usepackage{footnotehyper}}{\usepackage{footnote}}
\makesavenoteenv{longtable}
\usepackage{graphicx}
\makeatletter
\def\maxwidth{\ifdim\Gin@nat@width>\linewidth\linewidth\else\Gin@nat@width\fi}
\def\maxheight{\ifdim\Gin@nat@height>\textheight\textheight\else\Gin@nat@height\fi}
\makeatother
% Scale images if necessary, so that they will not overflow the page
% margins by default, and it is still possible to overwrite the defaults
% using explicit options in \includegraphics[width, height, ...]{}
\setkeys{Gin}{width=\maxwidth,height=\maxheight,keepaspectratio}
% Set default figure placement to htbp
\makeatletter
\def\fps@figure{htbp}
\makeatother
% definitions for citeproc citations
\NewDocumentCommand\citeproctext{}{}
\NewDocumentCommand\citeproc{mm}{%
  \begingroup\def\citeproctext{#2}\cite{#1}\endgroup}
\makeatletter
 % allow citations to break across lines
 \let\@cite@ofmt\@firstofone
 % avoid brackets around text for \cite:
 \def\@biblabel#1{}
 \def\@cite#1#2{{#1\if@tempswa , #2\fi}}
\makeatother
\newlength{\cslhangindent}
\setlength{\cslhangindent}{1.5em}
\newlength{\csllabelwidth}
\setlength{\csllabelwidth}{3em}
\newenvironment{CSLReferences}[2] % #1 hanging-indent, #2 entry-spacing
 {\begin{list}{}{%
  \setlength{\itemindent}{0pt}
  \setlength{\leftmargin}{0pt}
  \setlength{\parsep}{0pt}
  % turn on hanging indent if param 1 is 1
  \ifodd #1
   \setlength{\leftmargin}{\cslhangindent}
   \setlength{\itemindent}{-1\cslhangindent}
  \fi
  % set entry spacing
  \setlength{\itemsep}{#2\baselineskip}}}
 {\end{list}}
\usepackage{calc}
\newcommand{\CSLBlock}[1]{\hfill\break\parbox[t]{\linewidth}{\strut\ignorespaces#1\strut}}
\newcommand{\CSLLeftMargin}[1]{\parbox[t]{\csllabelwidth}{\strut#1\strut}}
\newcommand{\CSLRightInline}[1]{\parbox[t]{\linewidth - \csllabelwidth}{\strut#1\strut}}
\newcommand{\CSLIndent}[1]{\hspace{\cslhangindent}#1}

\KOMAoption{captions}{tableheading}
\makeatletter
\@ifpackageloaded{caption}{}{\usepackage{caption}}
\AtBeginDocument{%
\ifdefined\contentsname
  \renewcommand*\contentsname{Table of contents}
\else
  \newcommand\contentsname{Table of contents}
\fi
\ifdefined\listfigurename
  \renewcommand*\listfigurename{List of Figures}
\else
  \newcommand\listfigurename{List of Figures}
\fi
\ifdefined\listtablename
  \renewcommand*\listtablename{List of Tables}
\else
  \newcommand\listtablename{List of Tables}
\fi
\ifdefined\figurename
  \renewcommand*\figurename{Figure}
\else
  \newcommand\figurename{Figure}
\fi
\ifdefined\tablename
  \renewcommand*\tablename{Table}
\else
  \newcommand\tablename{Table}
\fi
}
\@ifpackageloaded{float}{}{\usepackage{float}}
\floatstyle{ruled}
\@ifundefined{c@chapter}{\newfloat{codelisting}{h}{lop}}{\newfloat{codelisting}{h}{lop}[chapter]}
\floatname{codelisting}{Listing}
\newcommand*\listoflistings{\listof{codelisting}{List of Listings}}
\makeatother
\makeatletter
\makeatother
\makeatletter
\@ifpackageloaded{caption}{}{\usepackage{caption}}
\@ifpackageloaded{subcaption}{}{\usepackage{subcaption}}
\makeatother

\ifLuaTeX
  \usepackage{selnolig}  % disable illegal ligatures
\fi
\usepackage{bookmark}

\IfFileExists{xurl.sty}{\usepackage{xurl}}{} % add URL line breaks if available
\urlstyle{same} % disable monospaced font for URLs
\hypersetup{
  pdftitle={Supporting Information},
  hidelinks,
  pdfcreator={LaTeX via pandoc}}


\title{Supporting Information}
\author{}
\date{}

\begin{document}
\maketitle


\setstretch{1.5}
\textbf{Authors:}

A. Cebulski\textsuperscript{1} (ORCID ID - 0000-0001-7910-5056)

J.W. Pomeroy\textsuperscript{1} (ORCID ID - 0000-0002-4782-7457)

\textsuperscript{1}Centre for Hydrology, University of Saskatchewan,
Canmore, Canada

\textbf{Corresponding Author:} A. Cebulski, alexcebulski@gmail.com

\section{Detailed Description of UAV-Lidar
Methodology}\label{detailed-description-of-uav-lidar-methodology}

The REIGL miniVUX-2 laser operates at a near infrared wavelength with a
laser beam footprint of 0.160 m x 0.05 mm (at 100 m above ground). The
accuracy and precision of the miniVUX-2 is described by REIGL for a lab
environment of 0.015 m and 0.01 m respectively (at 50 m above ground).
The miniVUX-2 was configured with a laser pulse repetition rate of 200
kHz, field of view of 360°, scan speed of 31.09 revolutions
s\textsuperscript{-1} and an angular step width of 0.0558°, resulting in
an expected an average point cloud density of 107 returns
m\textsuperscript{-2} for each flight path.

Georeferenced point clouds with x, y, and z coordinates for each laser
return were generated following methods outlined by Harder et al. (2020)
and Staines \& Pomeroy (2023) to reconcile survey lidar, IMU and GNSS
data. A ground-based GNSS system was positioned on a permanent monument
during each survey and underwent precise point positioning (PPP)
correction by Natural Resources Canada (2024). Differential GNSS
correction of the UAV trajectory was conducted using the ground-based
PPP GNSS observations and the POSPac UAV software. The UAV-lidar point
clouds were then transformed from a sensor referenced coordinate system
to a georeferenced coordinate system (EPSG:32611 - WGS 84 / UTM zone
11N) using the RIEGL Riprocess Software. A vertical offset of up to 6 cm
between UAV-lidar flight lines was observed in the resulting point
clouds on March 13\textsuperscript{th} and 14\textsuperscript{th}, 2024
and was attributed to IMU position drift. This offset between flight
lines was corrected using the BayesStripAlign software v2.24 (BayesMap
Solutions, 2024), which reduces relative and absolute uncertainties in
the vertical elevation of the point cloud using the ground control
points (GCP) collected across the study site using a differential GNSS
rover.

Quality control, ground classification and calculation of the change in
between two UAV-lidar point clouds was conducted using the LAStools
software package (LAStools, 2024). The ground classification was
conducted using the ``lasground\_new'' function (LAStools, 2024) for
both the pre and post snowfall event point clouds, with a step size set
to 2 m and 8 substeps (ultra\_fine setting). The offset and spike
options were set to remove points that are more than 0.1 m above or
below the initial ground surface estimate surface which
``lasground\_new'' fits to the last returns. This function is based on
an algorithm outlined by Axelsson (2000), describing the process of
making the initial ground surface element.

The change in elevation between the two UAV-lidar surveys was
interpreted as the increase in snow accumulation, \(\Delta HS\) over the
snowfall event. This change was calculated using a point-to-grid
subtraction method, using the ``lasheight'' function from the LAStools
(2024) software, as in Deems et al. (2013) and Staines \& Pomeroy
(2023). The pre snowfall event point cloud from ``lasground\_new'' by
``lasheight'' to construct a ``ground'' TIN. Subsequently, the height of
each post snowfall event point above the ground TIN, resulting in a
point cloud representing \(\Delta HS\). This point cloud was then
converted into a raster of \(\Delta HS\) with a grid cell resolution of
5 x 5 cm using the ``las2dem'' function. Further quality control and
resampling of the 5 cm raster of \(\Delta HS\) was conducted using the
`Terra' R package (Hijmans, 2024). Areas that were disturbed over the
snowfall event during the in-situ snow survey and values that exceeded
the .999th quantile were removed. To help remove any remaining noise a
0.25 m \(\Delta HS\) raster was generated by computing the median of the
5 cm \(\Delta HS\) values within each 0.25 m grid cell.

A comparison of UAV-liar and in-situ snow survey measurements over the
March 13--14th snowfall event and associated error metrics are shown in
Figure~\ref{fig-lidar-vs-fsd-sd}.

\begin{figure}[H]

\centering{

\includegraphics{figs/external_figures/23_073_v2.0.0_sa_snow_depth_Hs_insitu_vs_Hs_lidar_resamp_bias_corrected.png}

}

\caption{\label{fig-lidar-vs-fsd-sd}UAV-liar and in-situ snow survey
measurements over the March 13--14th snowfall event and associated error
metrics.}

\end{figure}%

\section{Linear Regression Models Through the
Origin}\label{linear-regression-models-through-the-origin}

Kozak \& Kozak (1995) noted, the default \emph{R}\textsuperscript{2}
value provided for least squares models forced through the origin by
many statistical packages can be misleading. Therefore, these
\emph{R}\textsuperscript{2} values were adjusted using Equation 10 in
Kozak \& Kozak (1995) and two statistical tests as described by Kozak \&
Kozak (1995) were used to verify whether a no-intercept model (forced
through the origin) was appropriate for this data compared to a
with-intercept model. The first test evaluated if the intercept of the
with-intercept was significantly different from zero using p-value
provided by the `summary' function from the `stats' package in R (R Core
Team, 2024). The second test examined if there was a significant
difference between the no-intercept and with-intercept models by testing
if the residual sum of squares was different between the no-intercept
and full model, assessed via Equation 15 in Kozak \& Kozak (1995). If
the first test indicated a significant difference, and the second did
not, the no-intercept model could be deemed statistically justified
(Kozak \& Kozak, 1995).

\pagebreak

\section*{References}\label{references}
\addcontentsline{toc}{section}{References}

\phantomsection\label{refs}
\begin{CSLReferences}{1}{0}
\bibitem[\citeproctext]{ref-Axelsson2000}
Axelsson, P. (2000). {Peter Axelsson 110}. \emph{International Archives
of Photogrammetry and Remote Sensing}, \emph{33}(4), 110--117.
\url{https://www.isprs.org/proceedings/XXXIII/congress/part4/111_XXXIII-part4.pdf}

\bibitem[\citeproctext]{ref-BayesMap2024}
BayesMap Solutions. (2024). \emph{{BayesStripAlign}}.
\url{https://bayesmap.com/products/bayesstripalign/}

\bibitem[\citeproctext]{ref-Deems2013}
Deems, J. S., Painter, T. H., \& Finnegan, D. C. (2013). {Lidar
measurement of snow depth: A review}. \emph{Journal of Glaciology},
\emph{59}(215), 467--479. \url{https://doi.org/10.3189/2013JoG12J154}

\bibitem[\citeproctext]{ref-Harder2020}
Harder, P., Pomeroy, J. W., \& Helgason, W. D. (2020). {Improving
sub-canopy snow depth mapping with unmanned aerial vehicles: Lidar
versus structure-from-motion techniques}. \emph{Cryosphere},
\emph{14}(6), 1919--1935. \url{https://doi.org/10.5194/tc-14-1919-2020}

\bibitem[\citeproctext]{ref-Hijmans2024}
Hijmans, R. J. (2024). \emph{{terra: Spatial Data Analysis}}.
\url{https://cran.r-project.org/package=terra}

\bibitem[\citeproctext]{ref-Kozak1995}
Kozak, A., \& Kozak, R. A. (1995). {Notes on regression through the
origin}. \emph{Forestry Chronicle}, \emph{71}(3), 326--330.
https://doi.org/\url{https://doi.org/10.5558/tfc71326-3}

\bibitem[\citeproctext]{ref-LAStools2024}
LAStools. (2024). \emph{{Efficient LiDAR Processing Software (version
240220, academic)}}. \url{http://rapidlasso.com/LAStools}

\bibitem[\citeproctext]{ref-ppp2024}
Natural Resources Canada. (2024). \emph{{Precise point positioning}}.
\url{https://webapp.csrs-scrs.nrcan-rncan.gc.ca/geod/tools-outils/ppp.php}

\bibitem[\citeproctext]{ref-R2024}
R Core Team. (2024). \emph{{R: A Language and Environment for
Statistical Computing}}. R Foundation for Statistical Computing.
\url{https://www.r-project.org/}

\bibitem[\citeproctext]{ref-Staines2023}
Staines, J., \& Pomeroy, J. W. (2023). {Influence of forest canopy
structure and wind flow on patterns of sub-canopy snow accumulation in
montane needleleaf forests}. \emph{Hydrological Processes},
\emph{37}(10), 1--19. \url{https://doi.org/10.1002/hyp.15005}

\end{CSLReferences}




\end{document}
